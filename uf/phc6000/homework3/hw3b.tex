\documentclass{article}
\usepackage{multicol}
\usepackage{geometry}
\usepackage{endnotes}
\usepackage{color}
\usepackage{lipsum}
\let\footnote=\endnote
\def\enoteheading{}
\setlength{\footskip}{20pt}
\usepackage{fancyhdr}
\fancyhf{}
\fancyhead[C]{Joe Brew $\bullet$ UFID: 0402-8902 $\bullet$ joebrew@gmail.com}
\fancyfoot[C]{PHC 6000 $\bullet$ Homework 3  $\bullet$ Epidemiology Methods I}
\renewcommand\headrulewidth{1pt}
\renewcommand\footrulewidth{1pt}
\pagestyle{fancy}

\usepackage{Sweave}
\begin{document}
\Sconcordance{concordance:hw3b.tex:hw3b.Rnw:%
1 17 1 1 0 116 1}

\newgeometry{margin=2.5cm}
\fancyhfoffset[O,O]{0pt}

\begin{center}
\begin{huge}
Homework 3 \end{huge} \\
\textbf{Joe Brew}
\end{center}

\vspace{3mm}

\begingroup
\leftskip3em
\rightskip\leftskip
\noindent \emph{The following is a critical evaluation of O'Leary et al.'s} \normalfont{Shift Work, Light at Night, and Breast Cancer on Long Island, New York}.\endnote{O'Leary ES, Schoenfeld ER, Stevens RG, Kabat GC, Henderson K, Grimson R, Gammon MD, Leske CL. Shift Work, Light at Night, and Breast Cancer on Long Island, New York.  Am J Epidemiol 2006;164:358-366.} \emph{The format of this critique follows the grading rubric, with the following respective sections: rationale, background/literature, methods, ethics, results, conclusions, bias/confounding, strengths, impression and miscellaneous.}
\par
\endgroup
\vspace{10mm}
\openup 0.5em


\begin{multicols}{2}
\section*{Rationale}
\emph{Discuss the rationale for the study and potential impact on population health.}\\
\vspace{1mm}

The rationale for the study was to evaluate the association between breast cancer and circadian disruption, given the hypothesis that "light-at-night" disrupts normal melatonin regulation, which in turn could raise one's risk of breast cancer.  The potential impact on population health is great: breast cancer is the most common cancer among all women in the United States\footnote{U.S. Cancer Statistics Working Group. United States Cancer Statistics: 1999-2010 Incidence and Mortality Web-based Report. Atlanta (GA): Department of Health and Human Services, Centers for Disease Control and Prevention, and National Cancer Institute; 2013. Available at: http://www.cdc.gov/uscs.}, and in an era of increasing "circadian disruption" due to light exposure both at work and at home, understanding the causes of breast cancer could have significant public health implications.

\section*{Background / Literature}
\emph{Discuss the quality and nature of background provided (e.g. Is the foundation adequately laid for what is known/unknown?  Is recent, relevant literature utilized?)}\\
\vspace{1mm}

The introduction (pages 358-359) lays out in clear and sequential terms the background literature which justifies a study of this nature.  75\% of the references in this section are to papers from within a decade of the publication date, meeting reasonable standards for "recent."  Though the metabolic processes by which cancer is hypothesized to develop in those with irregular light exposure is only tangentially described, this is explicitly addressed by acknowledging that "evidence is inconsistent" and that the "limited epidemiologic evidence" is based on studies with "small numbers of cases" (pg. 359).
\section*{Methods}
\emph{Comment on study methods, including appropriateness of study design, data collection, and recruitement.}\\
\vspace{1mm}

This is a retrospective \textbf{case-control study}, which carries along with it the usual limitations of the case-control design, namely a high suspectibility to bias\footnote{Cole P.  Case-Control Studies in Statistical Methods in Cancer Research.  IARC Scientific Publications(32):1, p. 22}, an inability to generate incidence data\footnote{Lewallen S, Courtright P. Epidemiology in Practice: Case-Control Studies.  Community Eye Health. 1998; 11(28): 57-58} and an inability to sufficiently provide evidence of causation.  Though the combination of survey and labwork in follow-up suggests strong internal validity, the limited geographical scope of the study (participants came from only two counties in New York) raises questions as to the external validity.  \\

Study recruitement is adequate but imperfect in two ways: (1) controls of different age groups were selected by different means (random digit dialing vs. HCFA rosters) and (2) controls matched only on age (in five-year groups) and not other potentially relevant factors, such as family history of cancer.

\section*{Ethics}
\emph{Comment on ethical considerations (e.g. consent, confidentiality).}\\
\vspace{1mm}

The authors explicitly address obtaining informed consent in both the collection of questionnaires and blood/urine samples, as well as during the at-home interview.  These procedures were approved by review borads at participating institutions, but little detail is given on the procedures themselves (pg. 359).  The epilogue contains an explicit reference to the fact that no conflicts of interests were declared on the part of any authors (pg. 365).  Though one might assume that standards of confidentiality were applied given multiple review boards' approval of the study, this is not directly addressed in the paper (and should have been).

\section*{Results}
\emph{Comment on the main findings and presentation of the results (e.g. do results make sense?  Is information missing that you'd like to see?  Are tables clear?)}\\
\vspace{1mm}

The results are laid out in both textual and tabular form.  Given the complexity of multiple comparisons of multiple exposures among multiple groups, the authors are correct to "guide" the reader with explicit references to the tables and even the rows in question (pg. 362).  \\

The results are coherent and complete (given the parameters and objectives laid out regarding the study's goals).  Given their complexity, visuals (such as a graphic of odds ratios with 95\% confidence intervals) would have been a useful addition, allowing the reader to more quickly understand the dose-respone effect (or lack thereof) of light-at-night and shift work.  Visuals would also have the advantage of allowing the authors to reduce the 5 paragraphs of text in the "Results" section.


\section*{Conclusions}
\emph{Comment on conclusions in light of the study/results (e.g. are they logical?).}\\
\vspace{1mm}

The authors' conclusions adequately address both the conflicting / inconsistent nature of their results, as well as how their study fits into the overall picture.  The authors point out that their study "provide[s] support" for the hypothesis that light-at-night may be associated with breast cancer (pg. 364), but also qualify this association with the potential bias due to differential recall, genetics, and the less-than-comprehensive manner by which night-time shift work was measured (pg. 365).  The authors are appropriately modest in the implications of their findings, noting that they "[do] not support shift work as a factor that increases breast cancer" and that the potential association of breast cancer to light-at-night exposure may be due largely to "differential recall" among cases and controls (pg. 365).

\section*{Bias and Confounding}
\emph{Discuss potential sources of bias and confounding in the study and whether they were adequately controlled for by the investigators.  Suggest sources of bias and/or confounding that were overlooked and should have been addressed, providing examples of ways they could have been addressed.}\\
\vspace{1mm}

The potential for bias and confounding in this study is high.  The authors largely address this in their "Strengths and limitations" section, pointing out the following noteworthy areas in which bias or confounding may have tainted the results of the study:\begin{enumerate}
\item The imprecise and potentially biased responses due to retrospective assessment of exposure (which are plausibly influenced by one's diagnosis status)
\item Incertainty regarding the natural history of disease, and the difficult balance between relatively short periods of exposure assessment for accurate recall (5-15 years) or long periods in order to include the "etiologically relevant" exposure period
\item The \textbf{selection effect} associated with the fact that long-term residency meant the de facto exclusion of many non-white, younger, pre-menopausal, non-drinking, non-hormone-taking educated people.  
\end{enumerate}

The above are noteworthy limitations, particularly selection effect, given that long-term residents may be different from more transient residents in many other (non-measured) ways.  \\

Finally, though the authors do not address it in depth, the circumstances which lead one to take upon night-time shift work, or to wake regularly at night, are likely confounded with potential risk factors for cancer.  "Adjusting" for income is insufficient in terms of the health effects of inequality, stress, and work type.\footnote{Marmot, M. G.; Rose, G.; Shipley, M.; Hamilton, P. J. (1978). "Employment grade and coronary heart disease in British civil servants". Journal of Epidemiology and Community Health 32 (4): 244-249}.  And adjustment for family history of cancer, parity and education is insufficient in terms of other socioeconomic, cultural, nutritional, exercise-related and other potential differential exposures to risk factors among those with different kinds of work and sleeping habits. \\

Given the design, further adjustment may have been unreasonable.  Nonetheless, bias and confounding are \emph{major} obstacles to the interpretation of this study's results.

\section*{Strengths and weaknesses}
\emph{Identify and discuss at least two strengths and two limitations of the study.}\\
\vspace{1mm}

\subsection*{Strength: uniqueness of study}
At the time of publication, this study was one of only two studies which examined the potential association between light-at-night exposure and breast cancer (pg. 364).  Furthermore, it is the only study of its type with such a strict residential limitation, a factor which has a negative impact on the study's generalizability, but likely goes a long way in reducing the potential differences between exposure groups (pg. 364).

\subsection*{Strength: gradient of measurements}
The authors astutely measured exposure in ways which went beyond the typical binary "exposed" vs. "non-exposed."  Shift work was measured by time of day, frequency and duration.  Light-at-night was measured by time period (peak vs. non-peak sleep), frequency and type of exposure (sleeping vs. non-sleeping).

\subsection*{Limitation: confounding}
A major limitation of this study is the fact that it does not fully account for the \textbf{why} of light-at-night exposure.  It is reasonable to assume that either psycological or physical factors determine one's predisposition to be exposed to light at night, and these factors may be correlated with cancer risk.  In other words, this exposure of interest to this study (light-at-night) may simply be a proxy for a more relevant risk factor for breast cancer.

\subsection*{Limitation: study design}
A second important limitation to this study is its design.  As a retrospective case-control study, it can neither assess exposure accurately (susceptible to recall bias), nor can it demonstrate causation (even if its results had been more consistent).  Of great importance to the particular domain of this study (the epidemiology of breast cancer), the case-control design cannot account for the temporal dimension of exposure.

\section*{Impression}
\emph{Provide thoughts on the overall quality and presentation of the study.}\\
\vspace{1mm}

My overall impression from this study is positive but qualified.  On a positive note, the sample size is large, the controls are fairly well-matched by age (and restricted by residency), and the majority of the limitations are explicitly and appropriately addressed by the authors.  That said, the potential for confounding is great, both in work type and in residential light-at-night exposure.  The authors appropriately adjusted odds ratios for five relevant variables (pg. 362), but neglected to account for many other potentially relevant exposures such as (a) occupational chemical exposure (particularly in the case of night-shift workers), (b) overweight / physical exercise, as well as others.  The study's findings, therefore, are inconsistent given the parameters they chose to measure, and even more questionable in light of exposures which were not addressed.\\

This study adds important input to the conversation on light-at-night, shift work and breast cancer.  The fact that the design was imperfect and that the results were inconsistent by no means should undermine the importance of this study.  This study constitutes an important contribution, particularly given that the literature on melatonin, light-at-night and cancer is sparse.

\section*{Miscellaneous}
\emph{Flow, format, proofreading, citations, etc.}\\
\vspace{1mm}

The \textbf{flow} of the paper is good and sequentially logical, going from justification and plausibility of study hypothesis to methods, results and discussion, while also addressing limitations along the way.  The \textbf{format} could have been improved by supplementing the tables with a chart of odds ratios, making the information more intuitively understandable and potentially reducing the need for so much text.  \textbf{Proofreading} appears fully adequate, as I found no errors despite multiple close readings.  The \textbf{citations} were well-referenced, relevant and ample enough to make a strong case for how this study fits into the big picture and why it is justified.  

\end{multicols}

\openup -01em
\section*{References}
\theendnotes

\end{document}
